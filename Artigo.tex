\documentclass[12pt]{article}
\usepackage{sbc-template}
\usepackage{graphicx,url}
\usepackage[utf8]{inputenc}
\usepackage[brazil]{babel}
\usepackage[latin1]{inputenc} 
\usepackage{hyperref}
\usepackage{natbib}

     
\sloppy

\title{Arquitetura para Análise da Prevalência de Autismo nos EUA com Dados Sintéticos em Tempo Real}

\author{Magna Fernandes\inst{1}, Bruno Schenberg\inst{1}, Rafael Colen\inst{1}, Renato Godoi\inst{1}}

\address{Universidade Presbiteriana Mackenzie (UPM)\\
CEP: 01302-907 -- São Paulo -- SP -- Brasil\\
\parbox{\textwidth}{\centering
\href{mailto:magna.fernandes@mackenzista.com.br}{magna.fernandes@mackenzista.com.br}, 
\href{mailto:bruno.schenberg@mackenzista.com.br}{bruno.schenberg@mackenzista.com.br}, 
\href{mailto:rafael.colen@mackenzista.com.br}{rafael.colen@mackenzista.com.br}, 
\href{mailto:renato.godoi@mackenzista.com.br}{renato.godoi@mackenzista.com.br}}
}

\begin{document} 

\maketitle
     
\begin{resumo} 
  Este artigo descreve uma arquitetura para análise da prevalência de autismo nos EUA utilizando dados da API do CDC.gov, enriquecidos com séries históricas e dados sintéticos gerados em tempo real. A arquitetura emprega Docker Compose, MongoDB, Kafka, Spark e Grafana para ingestão, processamento, análise e visualização dos dados.
\end{resumo}


\section{Introdução}

O Transtorno do Espectro Autista (TEA) é uma condição neurológica que afeta a comunicação social e o comportamento. A prevalência de TEA nos EUA tem aumentado nos últimos anos, tornando crucial a análise de dados para entender as tendências e desenvolver intervenções eficazes. Este trabalho propõe uma arquitetura para coletar, enriquecer e analisar dados de prevalência de autismo, utilizando tecnologias como Docker Compose, MongoDB, Kafka, Spark e Grafana, além de simular um cenário de inclusão de dados em streaming, o que seria extremamente benéfico para sociedade, ter o dado carregado NRT (near real time).

\section{Arquitetura Proposta} \label{sec:firstpage}

A arquitetura proposta é composta pelos seguintes módulos:

\subsection{Ingestão de Dados}
\begin{itemize}
    \item \textbf{API do CDC.gov}: Fonte de dados sobre a prevalência de autismo nos EUA.
    \item \textbf{Docker Compose}: Utilizado para orquestrar os serviços da arquitetura, incluindo a API, o MongoDB e o Kafka.
    \item \textbf{MongoDB}: Banco de dados NoSQL para armazenar os dados brutos da API e os dados enriquecidos.
\end{itemize}

\subsection{Enriquecimento de Dados}
\begin{itemize}
    \item \textbf{Série Histórica}: Dados de anos anteriores obtidos da API do CDC.gov ou de outras fontes confiáveis.
    \item \textbf{Dados Sintéticos}: Gerados em tempo real pelo Kafka Producer para simular a inclusão de novos dados e testar a capacidade de resposta da arquitetura.
    \item \textbf{Kafka}: Plataforma de streaming para o fluxo de dados sintéticos.
\end{itemize}

\subsection{Processamento e Análise}
\begin{itemize}
    \item \textbf{Spark}: Framework para processamento distribuído de grandes volumes de dados. Realiza a limpeza, transformação e agregação dos dados. Aplica algoritmos de análise estatística para identificar tendências e padrões na prevalência de autismo.
\end{itemize}

\subsection{Visualização}
\begin{itemize}
    \item \textbf{Grafana}: Ferramenta de visualização de dados para criar dashboards interativos com gráficos e métricas sobre a prevalência de autismo.
    \item \textbf{API de Resultados}: Expõe os dados processados e as análises para consumo do Grafana.
\end{itemize}


\section{Implementação}

A implementação da arquitetura foi realizada utilizando o Docker Compose, garantindo a fácil configuração e o gerenciamento de todos os componentes da solução. Cada componente é encapsulado em uma imagem Docker, garantindo portabilidade e reprodutibilidade. O código fonte pode ser encontrado no repositório https://github.com/mack-ppgca-asd-data/Autism-Data-Project---Mackenzie-2024.git.

\subsection{Imagens Docker}
\begin{itemize}
    \item \textbf{api-cdc}: Contém o código para consumir a API do CDC.gov e salvar os dados no MongoDB.
    \item \textbf{mongodb}: Instância do MongoDB para persistência dos dados.
    \item \textbf{kafka}: Instância do Kafka para o fluxo de dados sintéticos.
    \item \textbf{spark}: Instância do Spark para processamento e análise dos dados.
    \item \textbf{grafana}: Instância do Grafana para visualização dos dados.
\end{itemize}
\textbf{Docker Compose}: Define os serviços, as redes e os volumes da aplicação, facilitando a orquestração e o gerenciamento dos containers.

\section{Geração de Dados Sintéticos}

O Kafka Producer, emprega um processo contínuo utiliza a biblioteca Faker para gerar dados sintéticos em streaming, simulando a inclusão de novos casos de autismo. Esses dados são baseados em distribuições estatísticas e padrões observados nos dados reais, permitindo testar a capacidade da arquitetura de processar e analisar dados em tempo real. Esses dados são publicados em um tópico Kafka chamado "faker-data".

\section{Processamento com Spark}

O Spark processa os dados brutos da API, os dados históricos e os dados sintéticos. As etapas de processamento incluem:
\begin{itemize}
    \item \textbf{Extração de dados}: Extrai dados sobre autismo e informações demográficas de um banco de dados MongoDB.
    \item \textbf{Transformação de dados}: Limpa e formata os dados, incluindo a remoção de campos desnecessários e conversão de tipos de dados.
    \item \textbf{Enriquecimento}: Enriquece os dados adicionando nomes de estados com base em códigos FIPS.
    \item \textbf{Agregação de dados}: Agrega dados por estado, incluindo população dos EUA e população equivalente do Brasil.
    \item \textbf{Carga de Dados}: Carrega os dados transformados de volta para uma nova coleção no MongoDB.
\end{itemize}
Em resumo, o código extrai dados brutos sobre autismo, realiza transformações para torná-los mais informativos e úteis, e então os carrega em um banco de dados para análise posterior ou visualização.

\section{Visualização com Grafana}

O Grafana é utilizado para criar dashboards interativos que apresentam as análises e métricas sobre a prevalência de autismo. Os dashboards permitem visualizar:
\begin{itemize}
    \item \textbf{Tendências históricas}: Evolução da prevalência ao longo dos anos.
    \item \textbf{Distribuição geográfica}: Prevalência por estado ou região.
    \item \textbf{Dados demográficos}: Prevalência por idade e sexo.
    \item \textbf{Comparação com dados sintéticos}: Avaliação do impacto de novos dados na análise.
\end{itemize}

\section{Conclusões}

A arquitetura proposta oferece um framework flexível e escalável para análise da prevalência de autismo nos EUA. A combinação de dados reais em batch e sintéticos em NRT, junto com a utilização de ferramentas como Kafka, Spark e Grafana, possibilita estudos e análises mais abrangentes e relevantes, contribuindo para a compreensão e tratamento do autismo.

\section{Trabalhos Futuros}

\begin{itemize}
    \item Integrar outras fontes de dados, como informações socioeconômicas e ambientais, para aprofundar a análise da prevalência de autismo.
    \item Implementar modelos de Machine Learning para previsão da prevalência e identificação de fatores de risco.
    \item Desenvolver interfaces interativas para facilitar a exploração dos dados e a geração de insights.
\end{itemize}



% \section{Sections and Paragraphs}

% Section titles must be in boldface, 13pt, flush left. There should be an extra
% 12 pt of space before each title. Section numbering is optional. The first
% paragraph of each section should not be indented, while the first lines of
% subsequent paragraphs should be indented by 1.27 cm.

% \subsection{Subsections}

% The subsection titles must be in boldface, 12pt, flush left.

% \section{Figures and Captions}\label{sec:figs}


% Figure and table captions should be centered if less than one line
% (Figure~\ref{fig:exampleFig1}), otherwise justified and indented by 0.8cm on
% both margins, as shown in Figure~\ref{fig:exampleFig2}. The caption font must
% be Helvetica, 10 point, boldface, with 6 points of space before and after each
% caption.

% \begin{figure}[ht]
% \centering
% \includegraphics[width=.5\textwidth]{fig1.jpg}
% \caption{A typical figure}
% \label{fig:exampleFig1}
% \end{figure}

% \begin{figure}[ht]
% \centering
% \includegraphics[width=.3\textwidth]{fig2.jpg}
% \caption{This figure is an example of a figure caption taking more than one
%   line and justified considering margins mentioned in Section~\ref{sec:figs}.}
% \label{fig:exampleFig2}
% \end{figure}

% In tables, try to avoid the use of colored or shaded backgrounds, and avoid
% thick, doubled, or unnecessary framing lines. When reporting empirical data,
% do not use more decimal digits than warranted by their precision and
% reproducibility. Table caption must be placed before the table (see Table 1)
% and the font used must also be Helvetica, 10 point, boldface, with 6 points of
% space before and after each caption.

% \begin{table}[ht]
% \centering
% \caption{Variables to be considered on the evaluation of interaction
%   techniques}
% \label{tab:exTable1}
% \includegraphics[width=.7\textwidth]{table.jpg}
% \end{table}

% \section{Images}

% All images and illustrations should be in black-and-white, or gray tones,
% excepting for the papers that will be electronically available (on CD-ROMs,
% internet, etc.). The image resolution on paper should be about 600 dpi for
% black-and-white images, and 150-300 dpi for grayscale images.  Do not include
% images with excessive resolution, as they may take hours to print, without any
% visible difference in the result. 

\section{Referências}

\bibliographystyle{plainnat} % Escolha um estilo de bibliografia
\bibliography{Template_SBC/template-latex/sbc-template} % Nome do arquivo .bib

\cite{cdc_data}
\cite{kafka}
\cite{spark}
\cite{grafana}
\cite{mongodb}


\end{document}
